\chapter{P2P Content Delivery}
\section{Introduction}

P2P applications such as Napster, Gnutella, FastTrack, BitTorrent, Skype and PPLive, have attracted the end users.
According to the P2P paradigm, the P2P network is formed by peers that equally share the computing resources in a cooperative manner.
Each peer contributes of its resources such as network bandwidth, storage, etc. 
As the nodes of P2P grows, the capacity of the network grows, too because many peers join to the network. 
This enables a P2P application is cheap and it can be used for content delivery.  

We noted that the popular P2P file sharing applications are Gnutella, eDonkey, and Bittorrent.  
Gnutella is one of the earliest P2P file sharing applications in the Internet.
Gnutella is pure P2P applications that do not use a centralized server.
A Gnutella peer joins the network via at least one known peer.
That known peer IP address is obtained via pre-configured addresses. 
From this initial Gnutella peer, we can discover new Gnutella peers.
The Gnutella peer will send the search request to all connected peers. 
The recipient peer will answer the query if it knows anything. 
The recipient peer can forward the request to other peers if it does not know the answer.
Thus, The query will propagate among the Gnutella network.
Therefire Gnutella will floods the network for searching.

Other P2P applications is eDonkey.  
This applications is very popular in Europe. 
The eDonkey network operates as a hybrid P2P and server. 
The eDonkey network consists number of of clients and number of servers. 
IP address of server is usually pre-configured when user installing the applications for the first time.  
If users want to change the server, user can read on eDonkey web portal.
The eDonkey server is working as a indexes files and for distributing IP addresses of other eDonkey peers to the eDonkey users.

\begin{figure}[tb]
\begin{center}
\includegraphics[scale=0.6]{../../../papers/p2p-cdn-latex/tex/ipsj/graphs/how-bittorrent-works.eps}
\end{center}
\caption{Building block of Bittorrent. 
L term refer to Leecher and S term refer to Seeder. 
In a very popular torrent swarm, it is very common that seeder can be more than 40000 seeders and leecher can be more than 50000.
While 'get torrent' and 'fetch torrent file' only happen in the beginning of process when a peer join to Bitorrent swarm, the announce step will be repeated for every 30 minutes per Bittorrent specification.} 
\label{fig:bittorrentblock}
\end{figure}

BitTorrent was created in 2002 by Bram Cohen. 
It runs on an open protocol specification thus there a lot of Bittorrent implementation.
To share a file or a set of files through BitTorrent, a torrent file must be created for the first time.  
The torrent file contains the information of the content, which includes the information of the tracker and the hashes of the file blocks to be distributed.
The torrent file is usually distributed via Bittorent web portals.
When a client wants to get a file that shared in Bittorrent web portals, it must obtain the torrent file from Bittorrent portals.
The client then contacts the trackers listed in the torrent file to obtain a list of peers that are sharing the file at the same time.  
The announce step as shown in Fig.\ref{fig:bittorrentblock} is also a tool for peer to get current state of tracker e.q list of seeders and leechers.
Bittorrent specification defines minimum announce interval 30 minutes \cite{torrentspec}.
This process from initial phase to join Bittorrent swarm is shown in Fig.\ref{fig:bittorrentblock}
BitTorrent has generated great enthusiasm for P2P file sharing distribution due to simplicity.  
Many open source software projects use Bittorrent to distribute their applications which is quite big enough (equal or than CD disk space).
Important terms in Bittorrent are a seeder is a client who has complete chunks/blockss and a leecher is a client who download the chunks/blocks.

Beside of P2P for filesharing, we also noted that P2P for streaming is also on the rise especially in China.
The target of P2P streaming is to build a scalable P2P platform for TV/music delivery. 
More than a dozen companies are actively working in this area for example UUSe, PPLive, PPStream.  
Let's take PPLive as an example since it is the most popular P2P streaming service.
According to \cite{4378423} in 2007, the number of concurrent users for the most popular PPLive session raises to $1.5$ million and aggregate bandwidth up to $100$Gb/s.
When the PPLive client is launched by users, it retrieves from a channel servers the metadata information of all channels. 
After users choose the channel, the PPLive client further talks to a tracker of that channel. 
Next the trackers will give  a list of peers that are watching the same channel. 
After getting a list of peers, the PPLive client connects to a set of peers, and starts to exchange data. 
The challenge in P2P streaming is to provide a sustained streaming bit rate to all peers joining the network. 
In P2P streaming available bandwidth is very important.  
Insufficient bandwidth can cause poor video quality.  

\begin{figure}[tb]
\begin{center}
\includegraphics[scale=0.7]{../../../papers/p2p-cdn-latex/tex/ipsj/graphs/p2p-technologies.eps}
\end{center}
\caption{P2P Technologies Building Block. This graph is adapted from \cite{Passarella20121} with modification.} 
\label{fig:p2ptech}
\end{figure}

Due to complexity of P2P, many building blocks of P2P teechnology that need to explore related to content delivery. 
Figure \ref{fig:p2ptech} shows the presentation of P2P technologies building block.
The bottom level provides the basic networking abstractions, i.e., the P2P overlay networks. 
The middle level provides additional P2P services for delivery and management (including searching)
Above them, we can see P2P applications which work on the top of previous blocks.
Additional overlay abstractions can be exist in the middle to help delivery process. 
For examples:  mesh structure and gossip service.
Finally, security, trust and cooperation are seen as cross-layer issues including the mismatches between P2P applications cooperation and ISPs.

All P2P networks run on top of the Internet. 
We often consider the P2P network as an overlay network on the top of Internet. 
We can classify overlay to: structured, unstructured, and hierarchical. 
In structured overlays, the network is formed in a particular structure. 
The function of structured overlay is for storing and locating objects based on defined virtual address space. 
The space is organized according to a given geometry which defines neighbourhood relationship between peers.
For example Pastry \cite{Rowstron:2001:PSD:646591.697650}, which organized peers in a logical ring and pastry's address is a position in the ring. 

In unstructured systems, the unstructured overlay do not form any a fixed structure in the network. 
Unstructured overlay networks is typically close to random graphs. 
This is due to nature of every P2P applications that choose neighbors in random.
In unstructured P2P networks, peers usually do have not the same role with other peers and a peer can choose any other peers as neighbor to the some degree of freedom.  
In hierarchical overlay network, the network is built in an unstructured architecture with superpeers.
The superpeers is forming the top level and ordinary peers the bottom level.
In hierarchical overlay network, peers are organized in different groups and each group can runs its own overlay. 
A top-level of overlay can be formed by one representative per group.
This architecture will flexible than conventional unstructured with superpeers as a group can form a different overlay network either structured or unstructured. 
Figure \ref{fig:hierarchicalp2p} shows conceptual of hierarchical P2P network. 

\begin{figure}[tb]
\begin{center}
\includegraphics[scale=0.7]{../../../papers/p2p-cdn-latex/tex/ipsj/graphs/hierarchical-p2p.eps}
\end{center}
\caption{Hierarchical P2P Network.} 
\label{fig:hierarchicalp2p}
\end{figure}



Searching for content in overlay network is one of the key services in P2P.
Index is a keyword in searching inside P2P network.
In P2P networks, indices can be centralized, localized, or distributed. 
In structured overlay network centralized indices typically rely on a unique entity.
In unstructured overlay network localized indices is usually use for searching. 
Therefore, a query looking for a particular content must be propagated among peers.
However, indexing does not solve content poisoning problem in P2P.

In structured overlay networks, due to fix structured form, this networks can provide the support for exact match queries.  
Let's take Distributed Hash Table (DHT) for example. 
It has key and address of available content therefore a search operation results is fast and efficient.

Replication is tied with P2P networks.  
Let's take example in Bittorrent.  
If a user download content in Bittorrent,  it means that user replicated the contents.  
The block of content that already downloaded replicated in his/her PC and that block can be downloaded by other peers that need.
Consistency is important in Bittorrent, that's why Bittorrent provide hash for every block for checking the right block and reject bad block from fake content.

Gossiping is a simple and effective mechanism to spread the information. 
In P2P overlay networks, each peer participating in a gossiping process contacts of other peers in random and exchange the information between them.
Gossiping can be in push or pull mode.
Most of P2P applications working on both modes.
In Bittorrent implementation gossiping is in push and pull mode and the set of peers to contact is chosen at random.

Another important aspect in security of P2P is related to trust, and particularly to establishment of reputation. 
Establishing reputation of peers requires collecting information about previous interactions. 
For example, Bittorrent tracker usually has log to record the peers behavior. 
If a peer has bad upload/download ratio, Bittorrent tracker can blacklist that peer to join the Bittorrent swarm since the tit-for-tat policy of Bittorrent is the basis of cooperation enforcement.
We introduce private Bittorrent swarm term where user can join Bittorrent swarms by paying a membership.  
In private Bittorrent swarm, the trust is very important. 
Any member who can not fulfill the requirement (e.g minimum upload andwidth, minimum uptime after finishing download) can be blocked from Bittorrent swarm.

We highlight an important aspect related to the impact of P2P technologies on ISP policies (cross layer issues). 
P2P solutions are network agnostic, in the sense that they do not take any consideration the layer 3 of the network paths. 
Having agnostic P2P solutions can cause problems at different levels to ISPs. 
It has been shown that blind selections of neighbors in P2P networks may result in unnecessary traversal of multiple links inside an ISP. 
Furthermore, it can significantly impact on the shape and amount of traffic among different ISPs, which may result being quite different from what foreseen in ISPs peering agreements.
Although, many solution to this problem are exist, the implementations are still not deployed widely.  

While algorithms and mechanism in P2P system has been analyzed by many researchers, measurement of P2P is other aspect that important to do. 
First example: in cross layer issue, P2P is often blind in selection of neighbors in P2P networks that affect underlying ISP's traffic engineering policies.  
This effect can only seen if we have good real measurement instead doing simulation.
Second example: in Bittorrent economic model, reseachers wants to understand the content publishing phenomena. 
The growing popularity of Bittorrent is primary due to the availability of valuable content without any cost for users.
However, publishing valuable content has legal implications for the users who publish the content.
This raise the question that if content publisher behave in altruistic manner or financial incentives. 
This question can only be answered by doing measurement in real Bittorrent swarm. 

Energy was another aspect that is quite separated from current P2P research but now that topic becomes to rise. 
While popularity of P2P is decline because users now the content to the cloud, in some region P2P filesharing (e.g. Bittorent) applications is still popular.
Even now content providers, CDN providers, and ISPs are actively exploring the use of peer-to-peer (P2P) technologies to distribute content to homes as a means of reducing both file download times and the energy consumption of data centers. 
This approach pushes the energy use out of the data centers and into the homes of content consumers including P2P based applications that run on mobile devices.
With million house are connected to the Internet and million mobile devices are also connected to the Internet, it is very important to consider energy consumption in P2P network.


\section{P2P Measurement}

Bittorrent is the most successful Peer-to-Peer (P2P) application and was responsible for a major portion of Internet traffic in the past.
This has attracted the interest of the research community to evaluated the performance and the demographic aspects of BitTorrent.
Example current popular an application that use Bittorrent like protocol is Spotify which is a very popular streaming music application in Europe and US.
Bittorent has been largely studied using simulations, models and real measurements. 
Although simulations and modelling are easier to perform, they typically simplify the problems and they are likely to miss some of the effects which occur in real Bittorent swarms. 
Several techniques have been used in order to measure different aspects of BitTorrent so far. 
Since many popular applications work based on Bittorrent like protocol these days, we will focus on it.

\subsection{Measuring BitTorrent}
In this sub section we describe the BitTorrent measurement techniques defined in the literature so far. 
We classify them into two main categories macroscopic and microscopic depending on the retrieved information. 
Table \ref{tab:measurementtechniques} shows the summary of different techniques on Bittorrent measurement. 

\begin{table}[thb]
\caption{Comparison of Measurement Techniques in Bittorrent.}
\label{tab:measurementtechniques}
\hbox to\hsize{\hfil}
\begin{tabular}{c|c|c|c}\hline\hline
Property & Portal crawling & Tracker crawling & Peers crawler \\ \hline
Scope & macroscopic & macroscopic & microscopic \\ \hline
Information level & torrents level & demographics and  & peer level \\
 &  & general performance  & performance \\ \hline
Cost of crawler preparation & low & medium & high \\ \hline
Obtained result details & basic & medium & high \\ \hline
Peers population results & - & high & very high \\ \hline
\end{tabular}
\end{table}

\subsubsection{Macroscopic Technique}
The goal macroscopic technique is to understand the demographics of the Bittorrent ecosystem for example the type of published content, the popularity of the content, the distribution of BitTorrent users per country, etc. 
The macroscopic measurement allows us to get the performance information such as the ratio of between seeder to leechers, the session time of the Bittorrent users, the seedless state (period the torrent is without seeder) duration, etc.
We can classify the macroscopic techniques into two categories: Bittorrent portals crawling and Bittorrent trackers crawling.

\begin{itemize}
\item BitTorrent portals crawling: 
The (major) Bittorrent portals index millions of torrents in a structured way. 
They provide detailed information about each indexed torrent on a specific torrent web page. 
Once we know the how Bittorrent portals indexing the torrents, we can crawl the Bittorrent portals using that index.
If we want to analyze the demographics of Bittorrent, we need to crawl a large number of torrents.
This is take time since millions of torrents exist in Bittorrent portals, every second a new torrent can be published to Bittorrent portals (including fake torrent).
By processing the data from the Bittorrent crawling, we can get the information related to Bittorrent demographics.
For example: content popularity distribution, distribution of published content, and publishing rate of new torrents.
It's depends on Bittorrent web portals, some Bittorrent portals do not give detail information about the torrent information. 

\begin{figure}[tb]
\begin{center}
\includegraphics[scale=0.6]{../../../papers/p2p-cdn-latex/tex/ipsj/graphs/tracker-crawling.eps}
\end{center}
\caption{Bittorrent tracker crawling: The Bittorrent crawler fetch the torrent file from Bittorrent portals. Using information from torrent file, the Bittorrent crawler contacts the Bittorrent tracker by sending announce message. The Bitorrent tracker will reply by giving list of seeders and leechers. This process is repeated until obtain all the peers in the swarm.} 
\label{fig:trackercrawling}
\end{figure}


\item BitTorrent trackers crawling:
While crawling Bittorrent portals can give us information about the torrent type and torrent publisher and some number of seeders and leechers, that technique is not sufficient for more detail information such as distribution of users per country or performance relevant information such as peers arrival rate and peers session time.
To get that information, we need to collect IP address of seeders and leechers in the Bittorrent swarms.  
This information can only get from Bittorrent tracker logs by asking to Bittorrent tracker owners or by crawling the Bittorent tracker.
Figure \ref{fig:trackercrawling} shows the schematic of Bitorrent tracker crawling.  
First, a Bittorrent crawler needs to download torrent file from Bittorrent portal.  
Next, a Bittorrent crawler send announce message to Bitorrent tracker and Bitorrent tracker will reply by giving list of seeders IP address and list of leechers IP address.  
A Bittorrent crawler send announce message again to Bitorrent tracker. 
Since Bitorrent tracker has been recorded the IP address of the Bittorrent crawler, Bittorrent tracker will reply by giving list of leechers IP address only. 
This process repeat as many as possible to get list of leechers IP address.
If a Bittorrent crawler send too many announce message, Bittorrent tracker will block the Bittorrent crawler. 
Hence, this technique must be done in friendly way for Bittorrent tracker.
Another challenge is in Bittorrent the peers do not have a permanent peer-id.  
Everytime Bittorrent client is started a new random peer-id is generated thus it is very difficult to follow a peer using its peer-id. 
On the other since most of the Bittorrent client is home user which the IP address is assigned in a dynamic way by ISP, identifying peers by their IP address can introduce inaccuracies. 
\end{itemize}

\begin{figure}[tb]
\begin{center}
\includegraphics[scale=0.6]{../../../papers/p2p-cdn-latex/tex/ipsj/graphs/peer-crawling.eps}
\end{center}
\caption{Peer crawling: The initial process in same with the Bittorrent tracker crawling with additional steps for the Bittorrent crawler to contact all the peers in the swarm by sending handshakre request. The contacted peers will reply by sending BITFIELD message and HAVE message.} 
\label{fig:peeercrawling}
\end{figure}

\subsubsection{Microscopic Technique}
While in macroscopic techniques we can get the peers IP level information, that information does not sufficient to infer more detail performance metrics at the peer level such as peers download and upload rate, and how peers can connect each other (graph of the network).
To get this information, we need more sophisticated techniques that implement wire level Bittorent protocol since the crawling techniques in this case means join directly to Bittorrent swarm.
Figure \ref{fig:peeercrawling} shows the schematic functionality of peer crawling.  
In this peer crawling technique, a Bittorrent crawler get the IP address of peers participating in a swarm.
The initial process is same with tracker crawling. 
Afterwards, the Bitorrent crawler contact each peer and performs the handshake procedure. 
Handshake message must be the first message sent by Bittorrent client to other peers.
The contacted peers will reply by sending BITFIELD and HAVE message.
From handshake response we can get information if the peers is using public IP address space or behind NAT. 
Basically if contacted peers response to the Bittorrent crawler then that peers is using public IP address space and if that peers silent then that peers are behind NAT.
From BITFIELD message we can get information about peers type: seeder or leecher.   
By measuring the time between the reception of two consecutive HAVE messages from a peer, the Bittorrent crawler can calculate the time needed to download a chunk.
Chunk size information itself is always available in torrent file.
Thus, dividing the size of the chunk by the time need to download two consecutive HAVE message, we can infer the instantaneous download rate of a peer.
BITFIELD message contains information about chunks that already has in every peer in the swarms.  
Thus by comparing BITFIELD message for every peers in the swarm,  we can infer the chunk distributions in the swarms.
Challenges in this technique is the Bittorrent crawler can be blocked by a peer because it send many handshake message and receive many BITFIELD and HAVE message from a peer but the Bittorrent crawler does not upload any information to the peer.


\section{Energy Aspect of P2P Network}

Gupta and Singh \cite{Gupta:2003:GI:863955.863959} work on green networking has received a lot of attention.
In recent years, many efforts have been pushed to reduced energy expenditure.
Big companies and data center operator are trying new technologies to consume less energy.
For example Google which is planning to operate its data centers with a zero carbon footprint by using hydropower, water-based chillers, external cold air, etc for cooling its data center.
While energy is mostly concerned by companies that run hundred of server and hundred of networking devices inside data center, only a little attention have been made for home users. 
Remembering that P2P applications was the largest fraction of the Internet traffic in the past, although it is now decline due to the popularity of content sharing via cloud computing, we still have many popular P2P application that run based on P2P paradigm.  
For example, spotify.
Spotify is fee based music streaming service for mobile and desktop. 
It is very popular in Europe and USA. 
Spotify works like Bitorrent though there are several difference.
These figures shows that making P2P applications more energy efficient is important.
These days ISPs have evolved from providing basic Internet connectivity to offering higher valued services such as Internet+TV+phone.
In this value added service, the STBs or home gateways play an increasingly important. 
This STBs is same powerfull as PC desktop while the shape quite small compared to desktop PC. 
However, the powerfull of STB means consume more power. 
Since broadband home Internet users are increasing, consideration to reduce energy consumption of STB is required. 

Several approaches have been considered to reduce energy consumption. 
For example: Dynamic Voltage Scaling (DVS) and Dynamic Frequency Scaling (DFS) can be used to reduce energy consumption.
With DVS the supply voltage is reduced when not needed, hence we can get slower operation of circuit.
DFS can reduce the number of processor instructions, thus reducing performance.
Another approach is designing new network architectures.
For example: placing optical amplifiers at the most convenient locations and the task functions near renewable sources.
Performing resource consolidation, capitalizing on available energy is also considered as a way to reduce energy consumption.
This can be done via traffic engineering. 
Another way is by migrating computation or delegate the workload, typically using virtualization to move workloads transparently.

Due to the complexity of P2P networks, DVS and DFS can be implemented in hardware level.  
While in network level, migrating computation task is the another way to reduce energy consumption.  
For example in \cite{6249349}, Gianetti et al. propose the usage of Bittorrent proxy for  delegating the Bittorrent client work. 
Delegating some work to Bittorrent proxy can save up to $25\%$ of energy usage for PC desktop used for Bittorrent file sharing.
Delegating or proxying workload is not new idea.  
Example more primitive idea of proxying is Wake on LAN technology that introduced almost two decades ago \cite{wakeonlan}.
Wake on LAN technology allows computer to be awakened by \textit{magic} packet.

Since energy flow to consumers is based on supply and demand. 
The saving of small energy in home can affect distribution side of energy. 
This is known as cascade effect. 
It not only happened in home and its distribution but it is also appear in data center and its distribution.  

\begin{figure}[tb]
\begin{center}
\includegraphics[scale=0.8]{../../../papers/p2p-cdn-latex/tex/ipsj/graphs/cascade-energy.eps}
\end{center}
\caption{Cascade effetct:  1 watt energy saving in server component can reduce 2.8 watt energy in total facility level. This figure adapted from \cite{ciscocascade}} 
\label{fig:energycascade}
\end{figure}

While most network architecture energy consumption has been explored, a little work is done in exploring energy consumption in hybrid network architecture fashion. 
For example: energy consumption in combination of P2P and data center (CDN or Cloud).


\section{Related Work}

Bittorrent protocol performance has been explored extensively \cite{guo2005measurements}\cite{legout2006rarest}\cite{pouwelse2004measurement}\cite{tian2007modeling}\cite{li2010measurement}\cite{zhang2010bittorrent}.  
%The rarest first algorithm was discussed in \cite{legout2006rarest}, average download speed was discussed in \cite{pouwelse2004measurement}, peer arrival and departure process was discussed in \cite{guo2005measurements} and the effect of distributon of the peers on the download job progress was discussed in Y.Tian \textit{et al}. \cite{tian2007modeling}.
%The huge numbers of peers sending P2P download requests to random targets on the Internet and anti-P2P companies injecting bogus peers through PEX were discussed in Z.Li \textit{et al}. \cite{li2010measurement}.
%Higher upload-to-download ratios in  Bittorrent darknet were discussed in C.Zhang \textit{et al}. \cite{zhang2010bittorrent}.
Although we know that the topology can have a large impact on performance, to date only a few papers have addressed the issue.
Urvoy \textit{et al}. \cite{urvoy2007impact} used a discrete event simulator to show that the time to distribute a file in a Bittorrent swarm has a strong relation to the overlay topology.  
Al-Hamra \textit{et al}. \cite{al2007understanding}, also using a discrete event simulator, showed that Bittorrent creates a robust overlay topology and the overlay topology formed is not random. 
They also show that peer exchange (PEX) generates a chain-like overlay with a large diameter. 
Dale \textit{et al}. \cite{dale2008evolution}, in an experimental study on PlanetLab, show that in the initial stage of Bittorrent a peer will get a random peer list from the tracker. 
They found that a network of peers that unchoked each other is scale-free and the node degree follows a power-law distribution with exponent approximately 2. 
Dale \textit{et al}. \cite{dale2008evolution} also showed that the path length formed in Bittorrent swarms averages four hops and Bittorrent swarms have low average clustering coefficient.  
However, little work has been done on confirming that such controlled experiments correspond to the system. %that topology in the real world. 

We emphasize that compared to Ho\ss{}feld \textit{et al}. \cite{TR464}, our work provides a completely different approach and goal.
Ho\ss{}feld \textit{et al}. \cite{TR464} discuss the AS (Autonomous System) level topology of Bittorrent swarm for optimizing overlay traffic across ASes, while our study focus on microscopic dynamic aspect which is Bittorrent swarms topology itself (peer level or IP address level). 
The closest work to ours is Kryczka \textit{et al}. \cite{Kryczka2011}.
While our method is somewhat similar to theirs, they focus on clustering and locality while our focus is on node degree and clustering.
They use PEX to discover peer relationship, unfortunately they do not explain in detail how to process the PEX data set.
Because of differing PEX implementations between Bittorrent clients, we need to be careful with it in data processing.  
In our work, we describe PEX behavior and its limitation on two popular Bittorrent clients: Vuze and uTorrent, and we also explain how to treat PEX data from different Bittorrent clients. 
We also provide simulation result to confirm that our methods for infering peer relationship with PEX is valid.  
They observed that Bittorrent swarms have slightly higher clustering coefficient compared to random graphs of the same size and they observe neither Bittorrent swarm fulfills the properties of small world.
The slightly difference in clustering come from the difference of PEX data processing. 
They assume that PEX is the same and complete for all Bittorrent clients, therefore they get slightly different results. 

Our results agree with previous research \cite{dale2008evolution} in some areas and disagree in others, perhaps for two reasons.
First, power-law claims must be handled carefully. 
Many steps are required to confirm the power-law behavior, including alternative model checking, and we must be prepared for disappointment since other models may give a better fit. 
Second, our methodology relies on real work measurement combined with simulation for validation. 
This real-world measurement will reflect different types of clients connected to our swarm, and each client has a different behavior. 
%Each client might run on a different operating system, and of course clients are spread geographically. 
We also face difficult-to-characterize network realities such as NAT and firewalls. 
Our ability to reproduce key aspects of the topology dynamics suggests that these factors have only limited impact on the topology, somewhat to our surprise.

Content delivery networks (CDN) is a large distributed system of servers deployed in multiple data center across the Internet. 
The objetive of CDN is to serve content to end-users high availability and low latency manner as CDNs are distributed geographically. 
A CDN is owned, deployed, and maintained by a company that charges content providers or website owners for its services.
CDN with peer assist have been successfully deployed on the Internet, such as Akamai \cite{Huang:2008:UHC:1496046.1496064} and LiveSky \cite{Yin:2010:LEC:1823746.1823750}.  
The authors of \cite{Huang:2008:UHC:1496046.1496064} conclude from two real world traces that hybrid CDN-P2P can significantly reduce the cost of content distribution and can scale to cope with the exponential growth of Internet video content.  
Yin et al. \cite{Yin:2010:LEC:1823746.1823750} described LiveSkye as commercial operation of a peer-assisted CDN in China.  
LiveSky solved several challenges in the system design, such as dynamic resource scaling of P2P, low startup latency, ease of P2P integration with the existing CDN infrastructure, and network friendliness and upload fairness in the P2P operation.  
Measurement from LiveSky showed that LiveSky can save bandwidth around 40\% \cite{Yin:2010:LEC:1823746.1823750}.
The author in \cite{Huang:2007:IVP:1282427.1282396} and \cite{huang2007peer} proposed mechanisms to minimize CDN server bandwidth to make the content distribution cheap.
They designed different peer prefetching policies of video on demand system in surplus mode while ensuring user quality of experience.
A similar analysis has been done in \cite{xu2006analysis} for live video streaming system where the authors proposed different limited peer contribution policies to reduce CDN bandwidth requirement and eventually off the distribution process from CDN to P2P system. 
An ISP friendly rate allocation schemes for a hybrid CDN-P2P video on demand system in \cite{Wang:2008:IRA:1459359.1459397}. 
These technique try to minimize CDN server bandwidth while reducing ISP unfriendly traffic and maximizing peer prefetching.
Load on CDN server has been shown to be reduced using this approach while reducing cross ISP traffic.
Above studies were performed for video on demand or live video streaming.
While video is the most popular content, the systems can be also for other type contents.
Moreover while content based services are growing, energy consumption of a content distribution system has not been analyzed.

Related to CDN and energy usage, in a seminal work \cite{qureshi2009cutting}, the authors show that if costs are based on electricity usage and if the power prices vary in real-time, global load balancing decision can be made such that users are routed to locations with the cheapest power without significantly impacting user performance or bandwidth cost.  
In \cite{Palasamudram:2012:UBR:2391229.2391240}, the author proposed utilizing batteries for CDN for reducing total supplied power and total power costs.
The authors in \cite{Palasamudram:2012:UBR:2391229.2391240} also proposed battery provisioning algorithms based on workload of CDN server. 
The result shows that batteries can provide up to 14\% power savings.

The idea that utilize ISP controlled home gateway to provide computing and storage services and adopts managed peer-to-peer model is presented in \cite{valancius2009greening}. 
Valancius et al. \cite{valancius2009greening} show that distributing computing platform in NaDa (Nano Data Center) save at least 20-30\% energy compare to traditional data centers.
The saving in NaDa comes from underutilizing home gateway, avoidance of cooling costs, and the reduction of network energy consumption as a result of demand and service co-localization.

The comparison between CDN architecture and peer-to-peer architecture are discussed in \cite{baliga2007energy} and \cite{feldmann2010energy}.
Both authors in \cite{baliga2007energy} and \cite{feldmann2010energy} agree that CDN architecture is more energy saving compare to peer-to-peer architecture. 
Another interesting study of energy efficient in content delivery architectures is presented by Guan et al. \cite{5963557}.
Guan et al. \cite{5963557} compare energy efficient of CDN architecture and CCN architecture.
CCN is a new architecture to deliver the contents in the Internet \cite{Jacobson:2009:NNC:1658939.1658941}.  
CCN uses data storage cache at each level of the network e.g routers to decrease the transmission traffic and also increase the speed of response.
The authors in \cite{5963557} conclude that CCN is more energy efficient in delivering popular content while the approach with optical bypass is more energy efficient in delivering infrequent accessed content.

To the best of our knowledge, the study of energy in that considering peer-to-peer in CDN architecture is presented in \cite{6524219}.
Mandal et al. \cite{6524219} mentioned that hybrid CDN-P2P systems can reduce a significant amount energy in the optical core network around 20-40\% less energy.  
%The authors are focus on energy consumption of different algorithms for minimizing server bandwidth based on popularity of content, by maximizing peer request, the %authors can minimizing CDN workload. 
The authors only considered energy consumption of hardware especially optical devices and does not include overhead inside data center, CDN server energy comsumption, and consumed power by peers.
Our work is quite different, we take number of peers with static content and add overhead of data center which is  power of cooling cause by temperature of hardware for different purpose which are live streaming and video on demand.



