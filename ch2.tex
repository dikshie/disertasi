\chapter{Related Work}

Bittorrent protocol performance has been explored extensively \cite{guo2005measurements}\cite{legout2006rarest}\cite{pouwelse2004measurement}\cite{tian2007modeling}\cite{li2010measurement}\cite{zhang2010bittorrent}.  
%The rarest first algorithm was discussed in \cite{legout2006rarest}, average download speed was discussed in \cite{pouwelse2004measurement}, peer arrival and departure process was discussed in \cite{guo2005measurements} and the effect of distributon of the peers on the download job progress was discussed in Y.Tian \textit{et al}. \cite{tian2007modeling}.
%The huge numbers of peers sending P2P download requests to random targets on the Internet and anti-P2P companies injecting bogus peers through PEX were discussed in Z.Li \textit{et al}. \cite{li2010measurement}.
%Higher upload-to-download ratios in  Bittorrent darknet were discussed in C.Zhang \textit{et al}. \cite{zhang2010bittorrent}.
Although we know that the topology can have a large impact on performance, to date only a few papers have addressed the issue.
Urvoy \textit{et al}. \cite{urvoy2007impact} used a discrete event simulator to show that the time to distribute a file in a Bittorrent swarm has a strong relation to the overlay topology.  
Al-Hamra \textit{et al}. \cite{al2007understanding}, also using a discrete event simulator, showed that Bittorrent creates a robust overlay topology and the overlay topology formed is not random. 
They also show that peer exchange (PEX) generates a chain-like overlay with a large diameter. 
Dale \textit{et al}. \cite{dale2008evolution}, in an experimental study on PlanetLab, show that in the initial stage of Bittorrent a peer will get a random peer list from the tracker. 
They found that a network of peers that unchoked each other is scale-free and the node degree follows a power-law distribution with exponent approximately 2. 
Dale \textit{et al}. \cite{dale2008evolution} also showed that the path length formed in Bittorrent swarms averages four hops and Bittorrent swarms have low average clustering coefficient.  
However, little work has been done on confirming that such controlled experiments correspond to the system. %that topology in the real world. 

We emphasize that compared to Ho\ss{}feld \textit{et al}. \cite{TR464}, our work provides a completely different approach and goal.
Ho\ss{}feld \textit{et al}. \cite{TR464} discuss the AS (Autonomous System) level topology of Bittorrent swarm for optimizing overlay traffic across ASes, while our study focus on microscopic dynamic aspect which is Bittorrent swarms topology itself (peer level or IP address level). 
The closest work to ours is Kryczka \textit{et al}. \cite{Kryczka2011}.
While our method is somewhat similar to theirs, they focus on clustering and locality while our focus is on node degree and clustering.
They use PEX to discover peer relationship, unfortunately they do not explain in detail how to process the PEX data set.
Because of differing PEX implementations between Bittorrent clients, we need to be careful with it in data processing.  
In our work, we describe PEX behavior and its limitation on two popular Bittorrent clients: Vuze and uTorrent, and we also explain how to treat PEX data from different Bittorrent clients. 
We also provide simulation result to confirm that our methods for infering peer relationship with PEX is valid.  
They observed that Bittorrent swarms have slightly higher clustering coefficient compared to random graphs of the same size and they observe neither Bittorrent swarm fulfills the properties of small world.
The slightly difference in clustering come from the difference of PEX data processing. 
They assume that PEX is the same and complete for all Bittorrent clients, therefore they get slightly different results. 

Our results agree with previous research \cite{dale2008evolution} in some areas and disagree in others, perhaps for two reasons.
First, power-law claims must be handled carefully. 
Many steps are required to confirm the power-law behavior, including alternative model checking, and we must be prepared for disappointment since other models may give a better fit. 
Second, our methodology relies on real work measurement combined with simulation for validation. 
This real-world measurement will reflect different types of clients connected to our swarm, and each client has a different behavior. 
%Each client might run on a different operating system, and of course clients are spread geographically. 
We also face difficult-to-characterize network realities such as NAT and firewalls. 
Our ability to reproduce key aspects of the topology dynamics suggests that these factors have only limited impact on the topology, somewhat to our surprise.

Content Distribution Networks with peer assist have been successfully deployed on the Internet, such as Akamai \cite{Huang:2008:UHC:1496046.1496064} and LiveSky \cite{Yin:2010:LEC:1823746.1823750}.  
The authors of \cite{Huang:2008:UHC:1496046.1496064} conclude from two real world traces that hybrid CDN-P2P can significantly reduce the cost of content distribution and can scale to cope with the exponential growth of Internet video content.  
Yin et al. \cite{Yin:2010:LEC:1823746.1823750} described LiveSkye as commercial operation of a peer-assisted CDN in China.  
LiveSky solved several challenges in the system design, such as dynamic resource scaling of P2P, low startup latency, ease of P2P integration with the existing CDN infrastructure, and network friendliness and upload fairness in the P2P operation.  
Measurement from LiveSky showed that LiveSky can save bandwidth around 40\% \cite{Yin:2010:LEC:1823746.1823750}.
The author in \cite{Huang:2007:IVP:1282427.1282396} and \cite{huang2007peer} proposed mechanisms to minimize CDN server bandwidth to make the content distribution cheap.
They designed different peer prefetching policies of video on demand system in surplus mode while ensuring user quality of experience.
A similar analysis has been done in \cite{xu2006analysis} for live video streaming system where the authors proposed different limited peer contribution policies to reduce CDN bandwidth requirement and eventually off the distribution process from CDN to P2P system. 
An ISP friendly rate allocation schemes for a hybrid CDN-P2P video on demand system in \cite{Wang:2008:IRA:1459359.1459397}. 
Xu et al.\cite{DBLP:journals/corr/abs-1212-4915} used game-theory to show the right cooperative profit distribution of P2P can help the ISP to maximize the utility.  
Their model can easily be implemented in the context of current Internet economic settlements.  
Misra et al.\cite{Misra:2010:IPS:1811099.1811064} also mentioned the importance of P2P architecture to support content delivery networks.
The authors use cooperative game theory to formulate simple compensation rules for users who run P2P to support content delivery networks.
These technique try to minimize CDN server bandwidth while reducing ISP unfriendly traffic and maximizing peer prefetching.
Load on CDN server has been shown to be reduced using this approach while reducing cross ISP traffic.
Above studies were performed for video on demand or live video streaming.
While video is the most popular content, the systems can be also for other type contents.
Moreover while content based services are growing, energy consumption of a content distribution system has not been analyzed.

The idea of telco- or ISP-managed CDN has been proposed in recent years.  
The complexity of the CDN business encourage telcos and ISPs to manage their own CDN, rather than allow others to run CDNs on their networks.  
It has been shown that it is cost effective \cite{federation}\cite{norton2011internet}. 
Kamiyama et al. \cite{NoriakiKAMIYAMA2013} proposed optimally ISP operated CDN.
Kamiyama et al. mentioned that, in order to deliver large and rich Internet content to users, ISPs need to put their CDNs in data centers.  
The locations are limited while the storage is large, making this solution effective, using optimum placement algorithm based on real ISP network topologies.  
The authors found that inserting a CDN into an ISP's ladder-type network is effective in reducing the hop count, thus reduce total link cost.  
Based on the author definition: Ladder-type network is a network with a maximum degree under $10$.
Cisco has initiated an effort to connect telco- or ISP-managed CDNs to each other, to form a CDN federation \cite{federation} using open standards \cite{cdni}.  
They argue that the current CDN architecture is not close enough to the users and ISPs can fill this position.
Guo et al., \cite{1613869} work's PROP is closest with our work in peer-assisted CDN.
PROP uses local system (local counter) to calculate the segment popularity in peer-assisted proxy system. 
PROP uses popularity for proxy cache replacement strategy. 

The idea of utilizing the user's computation power to support ISP operation is not new.  
The Figaro project \cite{figaro} proposed the residential gateway as an integrator of different networks and services, becoming an Internet-wide distributed content management for a proposed future Internet architecture \cite{figaro}.  
Cha et al.,\cite{Cha:2008:NTP:1855641.1855646} performed trace analysis and found that an IPTV architecture powered by P2P can handle a much larger number of channels, with lower demand for infrastructure compared to IP multicast.  
Jiang et al. \cite{Jiang:2012:OMD:2413176.2413193} proposed scalable and adaptive content replication and request routing for CDN servers located in users' home gateways.  
Maki et al.,\cite{NaoyaMAKI2012} propose traffic engineering for peer-assisted CDN to control the behavior of clients, and present a solution for optimizing the selection of content files.
Mathieu et al., \cite{6249305} are using data gathered from France telecom network to calculate reduction of network load if customers are employed as peer-assisted co


Related to CDN and energy usage, in a seminal work \cite{qureshi2009cutting}, the authors show that if costs are based on electricity usage and if the power prices vary in real-time, global load balancing decision can be made such that users are routed to locations with the cheapest power without significantly impacting user performance or bandwidth cost.  
In \cite{Palasamudram:2012:UBR:2391229.2391240}, the author proposed utilizing batteries for CDN for reducing total supplied power and total power costs.
The authors in \cite{Palasamudram:2012:UBR:2391229.2391240} also proposed battery provisioning algorithms based on workload of CDN server. 
The result shows that batteries can provide up to 14\% power savings.

The idea that utilize ISP controlled home gateway to provide computing and storage services and adopts managed peer-to-peer model is presented in \cite{valancius2009greening}. 
Valancius et al. \cite{valancius2009greening} show that distributing computing platform in NaDa (Nano Data Center) save at least 20-30\% energy compare to traditional data centers.
The saving in NaDa comes from underutilizing home gateway, avoidance of cooling costs, and the reduction of network energy consumption as a result of demand and service co-localization.

The comparison between CDN architecture and peer-to-peer architecture are discussed in \cite{baliga2007energy} and \cite{feldmann2010energy}.
Both authors in \cite{baliga2007energy} and \cite{feldmann2010energy} agree that CDN architecture is more energy saving compare to peer-to-peer architecture. 
Another interesting study of energy efficient in content delivery architectures is presented by Guan et al. \cite{5963557}.
by Guan et al. \cite{5963557} comparing energy efficient of CDN architecture and CCN architecture.
The authors in \cite{5963557} conclude that CCN is more energy efficient in delivering popular content while the approach with optical bypass is more energy efficient in delivering infrequent accessed content.

To the best of our knowledge, the study of energy in that considering peer-to-peer in CDN architecture is presented in \cite{6524219}.
Mandal et al. \cite{6524219} mentioned that hybrid CDN-P2P systems can reduce a significant amount energy in the optical core network around 20-40\% less energy.  
%The authors are focus on energy consumption of different algorithms for minimizing server bandwidth based on popularity of content, by maximizing peer request, the %authors can minimizing CDN workload. 
The authors only considered energy consumption of hardware especially optical devices and does not include overhead inside data center, CDN server energy comsumption, and consumed power by peers.
Our work is quite different, we take number of peers with static content and add overhead of data center which is  power of cooling cause by temperature of hardware for different purpose which are live streaming and video on demand.



