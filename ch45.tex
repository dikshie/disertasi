\chapter{Peer-Assisted Content Delivery}

\section{Introduction}

Streaming content, especially video, represents a significant fraction of the traffic volume on the Internet, and it has become a standard practice to deliver this type of content using Content Delivery Networks (CDNs) such as Akamai and Limelight for better scaling and quality of experience for the end users. 
For example, YouTube uses Google cache and MTV uses Akamai in their operations.

With the spread of broadband Internet access at a reasonable flat monthly rate, users are connected to the Internet 24 hours a day and they can download and share multimedia content. P2P (peer to peer) applications are also widely deployed. 
In China, P2P is very popular; we see many P2P applications from China such as PPLive, PPStream, UUSe, Xunlei, etc. \cite{Vu:2010:UOC:1865106.1865115}. 
Some news broadcasters also rely on P2P technology to deliver popular live events. 
For example, CNN uses the Octoshape \cite{octoshape} solution that enables their broadcast to scale and offer good video quality as the number of users increases.

From the Internet provider point of view, the presence of so many always-on users suggests that it is possible to delegate a portion of computing, storage and networking tasks to the users, thus creating P2P networks where users can share files and multimedia content. 
Starting from file sharing protocols, P2P architectures have evolved toward video on demand and support for live events.

