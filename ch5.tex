\chapter{Green Networking}

Since the seminal paper by Gupta and Singh \cite{Gupta:2003:GI:863955.863959}, presented at SIGCOMM in 2003, the subject of green networking has received considerable attention. 
In recent years, valuable efforts have been devoted to reducing unnecessary energy expenditure.
Big companies such as Google, Microsoft, and Amazon, are turning to a host of new technologies to reduce operating costs and consume less energy.
Google, for example, is planning to operate its data centers with a zero carbon footprint by using, among other things, hydropower, water-based chillers, and external cold air to do some of the cooling.
Several approaches have been considered to reduce energy consumption in networks. These include:
\begin{itemize}
	\item The design of low power components that are still able to offer acceptable levels of performance. 
	For example, at the circuit level techniques such as Dynamic Voltage Scaling (DVS) and Dynamic Frequency Scaling (DFS) can be used. 
	With DVS the supply voltage is reduced when not needed, which results in slower operation of the circuitry. 
	DFS reduces the number of processor instructions in a given amount of time, thus reducing performance. 
	These techniques can reduce energy consumption significantly. 
	
	\item Consuming energy from renewable energy sources sites rather than incurring in electricity transmission overheads, thus reducing CO2 emissions.
	
	\item Designing new network architectures, for example by moving network equipment and network functions to strategic places. 
	Examples include placing optical amplifiers at the most convenient locations and performing complex switching and routing functions near renewable sources.
	
	\item Using innovative cooling techniques. Researchers in Finland, for instance, are running servers outside in Finnish winter, with air temperatures below $20$ degrees celsius.
	
	\item Performing resource consolidation, capitalizing on available energy. 
	This can be done via traffic engineering, for instance. 
	By aggregating traffic flows over a subset of the network devices and links allows others to be switched off temporarily or be placed in sleep mode. 
	Another way is by migrating computation, typically using virtualization to move workloads transparently.
	
\end{itemize}
