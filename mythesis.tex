%
% The thesis guidelines stipulate that "The two copies of the thesis for
% examination should be soft-bound (not ringbound) and printed on one side of
% the page only. It is required that the hard-bound copy of a thesis will be
% printed on both sides of the page on paper of a weight of at least 90 gsm"
% Therefore, use 
%
%   \documentclass[oneside]{tcd-phd-thesis}
%
% for the copy for examination, but
%
%   \documentclass[twoside]{tcd-phd-thesis}
%
\documentclass[oneside]{tcd-phd-thesis}
\usepackage{url}
\usepackage[dvips]{graphicx}
\usepackage[fleqn]{amsmath}
\usepackage[varg]{txfonts}
\usepackage{cite}
\usepackage{epsfig}
\usepackage{float}
\usepackage{caption}



\begin{document}

\input title

\frontmatter
%\input declaration
\input summary
\input acknowledgements

\mainmatter
\tableofcontents
\listoffigures

\input ch1
\input ch2
\input ch3
\input ch4
\input ch5
\input ch6
\input ch7


% .. other chapters go here ..

%
% If you have appendices, use
%
%   \appendix
% 
% and then \input your appendices the same way that you are inputting chapters,
% above
%

%
% "1.33.11 References. Systematic and complete reference to sources used and a
% classified list of all sources used must be included in the thesis. The
% titles of journals preferably should not be abbreviated; if they are,
% abbreviations must comply with an internationally recognised system (the
% format may vary according to the precedents and customs of the subject area;
% graduate students should consult with their Supervisor as to appropriate
% presentation)." 
%
% Replace "acm" by a different style for different styles of references (try
% "abbrv", "alpha", "apalike", "ieeetr", "plain" or "siam"). This affects the
% style of references within the text, the ordering of the bibliography, etc.
%

\bibliographystyle{acm}
\bibliography{references}

\end{document}
