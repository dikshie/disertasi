\begin{appendices}

\chapter{Some Detail on BitTorrent Terminology}

In this appendix, we will give more detail explanation some BitTorrent terminology based on BitTorrent specification. 

\section{Torrent File}
Torrent file or metainfo file has always ending in ".torrent".  
This file is a bencoded dictionary, containing structure as follows:
\begin{itemize}
	\item info: a dictionary described the file of the torrent. 
	\item announce: the announce URL of the tracker.
	\item creation date: the creation time of the torrent in standard UNIX epoch format. 
	\item comment: free text contain comments from the author.
	\item created by: name and version of the application used to create this torrent file or metainfo file.
	\item encoding: the string encoding format used to generate the pieces/chunks/blocks of the info dictionary in torrent file.
\end{itemize}
info itself has structure as follows:
\begin{itemize}
	\item piece length: the length of a piece/chunk/block in bytes. it is usually in power of 2. Common size for piece length is 256KB.
	\item pieces: concatenation of all 20 byte SHA1 hash value each pieces/chunks/blocks.
	\item private: this optional to denote private torrent.
\end{itemize}

\section{Handshake}.  
Handshake is the first message that must sent by BitTorrent client when contacting other peers. 
Handshake message format as follows: 
\begin{verbatim}
handshake: <pstrlen><pstr><reserved><info_hash><peer_id>
\end{verbatim}
The details as follows:
\begin{itemize}
	\item pstrlen is string length of <pstr>.
	\item pstr is string identifier of the protocol.
	\item reserved is reverse bit for changing the behavior of protocol.
	\item info\_hash is 20 byte SHA1 hash of the info key in torrent file. 
	\item peer\_id is 20 byte string used as a unique ID for the client.
	\item name:  the file name of content.
	\item length: length of the file in bytes.
	\item md5sum: hexadecimal string corresponding to the MD5 sum of the file. 
\end{itemize}

\section{HAVE}
HAVE message is the message that sent by a peer to other peers.  
The purpose of this message to let other peers know the pieces/chunks that has just downloaded.
The format of HAVE message as follows:
\begin{verbatim}
have: <len=0005><id=4><piece index>
\end{verbatim}
piece index is the index of a piece that has just successfully downloaded and verified by the hash.

\section{BITFIELD}

BITFIELD message is the message that sent by peer to other peers.
The purpose of this message to let other peers know total pieces/chunks that a peer have.
The format of BITFIELD message as follows:
\begin{verbatim}
bitfield: <len=0001+X><id=5><bitfield>
\end{verbatim}
The BITFIELD message has variable length because its depend on number of information of pieces/chunks that have been downloaded. 


\section{PEX}

There are two PEX implementation which are Azareus and Libtorrent (UT\_PEX).
We will use the UT\_PEX as reference. 
Since PEX is extension, to be able to use that extension, the reserved field in handshake message must be filled.
In this case, the 5th bit of the 6th byte of the reserved. 
Next, the handshake message informs other peers which extended messages are supported and what's the extended id will be used.   

The payload of a peer exchange message is a bencoded dictionary with the following keys:
\begin{itemize}
	\item added: contains list of peers in the compact tracker format since the last peer exchange message.
	\item added.f: one byte of flags for each peer in the above added string. This contains information about other supported things for UT\_PEX.  
	for example in UT\_PEX, byte 1 is for encryption thus the peer support encryption. 
	\item dropped:  contains list of peers that dropped since the last peer exchange message.
\end{itemize}


\chapter{Likelihood Ratio Test}
Assume we have two different distributions with PDFs $p_1(x)$ and $p_2(s)$.
The likelihood of a given data set within two distributions are 

\begin{align}
	L_1 &= \prod_{i=1}^{n} p_1(x_i)  &  L_2 &= \prod_{i=1}^{n} p_2(x_i) 
\end{align}

and the likelihood is: 
\begin{equation}
	R = \frac{L1}{L2} = \prod_{i=1}^{n} \frac{p_1(x_i)}{p_2(x_i)}
\end{equation}

taking log, the log likelihood ratio is:
\begin{equation}
	\mathcal{R} = \sum_{i=1}^{n} [ln \:  p_1(x_i) - ln \: p_2(x_i)] = \sum_{i=1}^{n} [ \ell_{i}^{(1)} - \ell_{i}^{(2)} ]
\end{equation}

where $ \ell_{i}^{(j)}$ is the log-likelihood in distribution $j$.
As we assume that $x_i$ are independent, thus $\ell_{i}^{(1)} - \ell_{i}^{(2)}$ is also independent. 
By the central limit theorem, their sum $\mathcal{R}$ will be in normal distribution as $n$ grows.
Expected variance can be estimate as:

\begin{equation}
	\sigma^2 = \frac{1}{n} \sum_{i=1}^{n}[(\ell_{i}^{(1)} - \ell_{i}^{(2)}) - (\bar{\ell}_{i}^{(1)} - \bar{\ell}_{i}^{(2)})]^2
\end{equation}
where 
\begin{align}
	\bar{\ell}^1 & = \frac{1}{n} \sum_{i=1}^{n} \ell_{i}^{(1)} & \bar{\ell}^2 & = \frac{1}{n} \sum_{i=1}^{n} \ell_{i}^{(2)}
\end{align}
The probability that the measured of log-likelihood ratio has magnitude as large as or larger than observed value $|\mathcal{R}|$ is: 

\begin{equation}
 p = erfc (\frac{|\mathcal{R}|} { \sqrt{2n\sigma}})
\end{equation}
and 
\begin{equation}
 erfc(z) = 1 - erf(z)
\end{equation}

is the complementary Gaussian error function which is available widely in scientific computing library.
This $p$ value is also call significance value for log-likelihood test. 
In nested hypothesis where we compare two distributions under same family, if $p$-value is small, the smaller family is better than larger family.
if not then we can not say no evidence that larger family is needed, while the smaller family gives better fit. 

\end{appendices}
